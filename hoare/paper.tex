\documentclass[twoside,twocolumn]{article}

\usepackage{blindtext} % Package to generate dummy text throughout this template 

\usepackage[sc]{mathpazo} % Use the Palatino font
\usepackage[T1]{fontenc} % Use 8-bit encoding that has 256 glyphs
\linespread{1.05} % Line spacing - Palatino needs more space between lines
\usepackage{microtype} % Slightly tweak font spacing for aesthetics

\usepackage[english]{babel} % Language hyphenation and typographical rules

\usepackage[hmarginratio=1:1,top=32mm,columnsep=20pt]{geometry} % Document margins
\usepackage[hang, small,labelfont=bf,up,textfont=it,up]{caption} % Custom captions under/above floats in tables or figures
\usepackage{booktabs} % Horizontal rules in tables

\usepackage{lettrine} % The lettrine is the first enlarged letter at the beginning of the text

\usepackage{enumitem} % Customized lists
\setlist[itemize]{noitemsep} % Make itemize lists more compact

\usepackage{abstract} % Allows abstract customization
\renewcommand{\abstractnamefont}{\normalfont\bfseries} % Set the "Abstract" text to bold
\renewcommand{\abstracttextfont}{\normalfont\small\itshape} % Set the abstract itself to small italic text

\usepackage{titlesec} % Allows customization of titles
\renewcommand\thesection{\Roman{section}} % Roman numerals for the sections
\renewcommand\thesubsection{\roman{subsection}} % roman numerals for subsections
\titleformat{\section}[block]{\large\scshape\centering}{\thesection.}{1em}{} % Change the look of the section titles
\titleformat{\subsection}[block]{\large}{\thesubsection.}{1em}{} % Change the look of the section titles

% \usepackage{fancyhdr} % Headers and footers
% \pagestyle{fancy} % All pages have headers and footers
% \fancyhead{} % Blank out the default header
% \fancyfoot{} % Blank out the default footer
% \fancyhead[C]{} % Custom header text
% \fancyfoot[C]{\thepage} % Custom footer text

\usepackage{dirtytalk}
\renewcommand{\quote}[2]{
  \say{
    {\itshape #1}
  }\\[.5em]
  \rightline{{\rm --- #2}}

  \vspace{1em}
}

\usepackage{titling} % Customizing the title section

\usepackage{hyperref} % For hyperlinks in the PDF

%----------------------------------------------------------------------------------------
%	TITLE SECTION
%----------------------------------------------------------------------------------------

\setlength{\droptitle}{-4\baselineskip} % Move the title up

\pretitle{\begin{center}\Large\bfseries} % Article title formatting
\posttitle{\end{center}} % Article title closing formatting
\title{An Axiomatic Basis for Computer Programming: Review} % Article title
\author{%
\textsc{Gosha Krikun}\\[1ex] % Your name
\normalsize Innopolis University\\ % Your institution
\normalsize \href{mailto:g.krikun@innopolis.ru}{g.krikun@innopolis.ru} % Your email address
%\and % Uncomment if 2 authors are required, duplicate these 4 lines if more
%\textsc{Jane Smith}\thanks{Corresponding author} \\[1ex] % Second author's name
%\normalsize University of Utah \\ % Second author's institution
%\normalsize \href{mailto:jane@smith.com}{jane@smith.com} % Second author's email address
}
% \date{February 10, 2017} % Leave empty to omit a date
\date{\today}
\renewcommand{\maketitlehookd}{%
\begin{abstract}
  \noindent

  This is review on C.A.R. Hoare's paper, An Axiomatic Basis for Computer
  Programming. 
  
\end{abstract}
}

%----------------------------------------------------------------------------------------

\begin{document}

% Print the title
\maketitle

%----------------------------------------------------------------------------------------
%	ARTICLE CONTENTS
%----------------------------------------------------------------------------------------

\section{Introduction}

``People are not perfect'', this is what we learned so fourth. 
Every one could make a mistake. In some areas it could be unnoticed, but in
others it could cause serious consequences.

There are a lot of people with their needs. Majority of them lie outside
safety-critical sector, error and failures in which could leads to deaths. 
People won't spent big money for time of qualified developers to avoid minor
bugs. 

So nowadays main approach to find failures and errors is testing.
And I can't not mentioned my favorite quote about it:\\

\quote{
Program testing can be used to show the presence of bugs, but never to show their absence!
}{Edsger Wybe Dijkstra}

But more and more critical tasks entrust to computer systems. When errors and
failures cost increases, we will pay for diligence and accuracy as much as
needed.

To be exactly sure, that we don't make mistakes we make premises and develop techniques for
reasoning about properties and correctness. And for proving it we used our logic
and deduction.  

\section{Purpose and Applications}

So Hoare suggested triple \{P\} Q \{R\} \cite{hoare:69} in which developer need
define 
precondition (premises) and postcondition (conclusion) which hold for program Q
after execution. 

This technique helps prove correctness of a program Q, according to defined
logical predicates P and R.

It doesn't help derive true predicates from requirements, but it
provide ability to use common proof techniques for deducting correctness.

For next disscussion suppose that we define right P and R from some
requirements. \\

Speaking about correctness we should understand, that any answer could be in
three state: answer received and it is correct, answer not received, answer is
incorrect.  

So when answer received and it is right, for all input values in the domain,
then we call it total correctness.

If answer could be not received (e.g. program doesn't terminate) for some values
in the domain, but is right for others - partial correctness.

And incorrectness in last case. \\

Using axioms and inference rules what was proposed by Hoare, we could deduct
that if P holds before execution of Q, and if Q terminates, then R will holds
after execution (correctness of Q).\\

In 1969 paper Hoare didn't mentioned that Q should terminates, so he defined
rules just for deducting partial correctness (particular in rule D3). But later
he add definition for total correctness too. \\

For example lets take task on finding real roots for quadratic equation:
$$ ax^2 + bx + c = 0$$

We limits input domain as set of real numbers and according to discriminant in
predicate P:
$$ P \equiv (a, b, c \in \mathbb{R}) \wedge (b^2 - 4ac \geq 0)$$

Then we could find values by some program Q. And finally check
whether equation holds or not in R. 

\section{Axioms and Rules}

In the article Hoare states four basic rules:
\begin{itemize}
\item
  Axiom of Assignment
\item
  Rule of Consequence
\item
  Rule of Composition
\item
  Rule of Iteration
\end{itemize}

Also it requires no side effects for Q (nothing changes except state which
we define in pre and postconditions). Because in other case, we couldn't
consider correctness of overall state. \\

Using this rules we can make deductive proof from our premise to conclusion,
that R will holds after execution. 

But only after, not during.

It could be a case when program Q during execution
change state which doesn't hold conditions P and R.

This case could appear when specification describe something in general, but
doesn't divide execution into small pieces of actions.

If we think about it in more details, we could decompose it to two sequence of
statements (rule of composition D2), in which first part hold some postcondition
R1 and second have R1 as precondition.

So nothing scary in that case if we still could prove correctness of our program.

P like a guard, filter input on what program will work correctly, and R check
results to be correct. 

% But sometimes it could be not so straightforward. Lets return to our example
% about real roots. Suppose quadratic coefficient a could be equals 0. 

% In that case equation is linear but still valid, there is solution when equation holds.
% But if we solve it by general formula (formula) 

% Rules.
% There is set of rules (or axioms) which we could use to deduct correctness of a program.


However precondition and postconditions could be almost same. This looks ugly.

firstly we should define the
truth. 

So directed tests could show that implementation isn't work properly. This is
helpful during development and evolving software product.


%------------------------------------------------



\section{Shortcomings}

\section{Conclusion}

\begin{thebibliography}{99} % Bibliography - this is intentionally simple in this template

\bibitem[Hoa69]{hoare:69}
C.~Antony~R. Hoare.
\newblock An axiomatic basis for computer programming.
\newblock {\em Communications of the ACM}, 12(10):576--580, 1969.

\end{thebibliography}

%----------------------------------------------------------------------------------------

\end{document}